\section{Описание}

\subsection{Общая концепция}

Эта аркадная игра представляет собой соревновательную и динамичную развлекательную платформу, где главная задача
игрока — набрать максимальное количество очков за ограниченное время.
Игрок метает мечи в различные цели, стремясь попасть точно и быстро, чтобы повысить свой счет и соревноваться с другими
игроками.

\subsection{Геймплей}

\begin{itemize}
	\item Метание мечей: Основное действие в игре — бросок меча в цель. Игрок управляет бросками, стремясь попадать точно в центр мишеней для максимального количества очков.
	\item Таймер: У игрока есть ограниченное количество времени на каждом уровне или раунде. Таймер отсчитывает оставшееся время, подгоняя игрока к быстрому и точному метанию.
	\item Очки: За каждое попадание в мишень игрок получает очки. Точные попадания (например, в центр мишени) приносят больше очков, чем попадания по краям.
	\item Сложность: По мере прогресса игрока, игра становится сложнее: мишени могут двигаться, уменьшаться в размерах или появляться в неожиданных местах.
\end{itemize}

\subsection{Основные особенности}
Игрок бросает меч по цели.
Игроку необходимо учитывать угол и силу броска, что требует навыка и точности.

\subsection{Визуальный стиль}

\begin{itemize}
	\item Интуитивно понятный: Минималистичный и удобный интерфейс, который не отвлекает от игрового процесса.
	\item Отображение таймера и очков: Важно, чтобы игрок всегда видел оставшееся время и текущий счет.


	\item Яркая и красочная: Использование ярких цветов и высококонтрастных элементов, чтобы привлечь внимание игрока.
	\item Простые и четкие линии: Мишени и интерфейс должны быть легко различимы, чтобы игрок мог быстро реагировать на изменения в игровом процессе.
	\item Анимация: Плавные и динамичные анимации при метании мечей, движении мишеней и начислении очков.
\end{itemize}



Эти элементы вместе создают захватывающий и визуально привлекательный игровой опыт, который будет удерживать
внимание игроков и побуждать их к новым попыткам побить свои рекорды и улучшить свои навыки.

