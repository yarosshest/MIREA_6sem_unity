\section{Разработка}

Разработка игры ведётся на движке Unity.
Для создания основных игровых механик,
таких как управление автомобилем, стрельба и генерация врагов,
используются скрипты на языке C\#.

\subsection{Управление }

Ниже представлен код, отвечающий за управление и игровой процесс:
\lstinputlisting[language=Python,
	caption=\leftline{Управление}]{code/control.cs}

Этот скрипт (\texttt{control.cs}) является основным для управления и игрового процесса.
Он обрабатывает создание мячей, подсчет времени, вывод очков.
Метод \textbf{Update()}
обрабатывает ввод от игрока, подсчет времени, вывод очков.

Код для отображения счета:
\lstinputlisting[language=C,
	caption=\leftline{Код для отображения счета}]{code/ScoreGUI.cs}


\subsection{Цели}

Код для цели:
\lstinputlisting[language=Python,
	caption=\leftline{Код для цели}]{code/Target.cs}

Скрипт \texttt{Target.cs} отвечает за префаб цели.
В методе \textbf{Start()} меняется текст в соответствии со значением цели.
Функция \textbf{OnTriggerEnter()} цель удаляется цель с мечем.

Код для генерации целей:
\lstinputlisting[language=Python,
	caption=\leftline{Код для генерации целей}]{code/TargetGen.cs}

Скрипт \texttt{TargetGen.cs} создание целей.
Метод \textbf{UpdateCylinders()} проверяет кол-во целей.
Метод \textbf{GenerateCylinders()} создает цилиндры.
Метод \textbf{GetRandomPosition()} создает точку внутри границ для генерации целей.
Метод \textbf{CheckValidPosition(Vector3 posNew)} проверяет точку на близкое расположение другим целям.
                                                                               
Код для удаления мячей:
\lstinputlisting[language=Python,
	caption=\leftline{Код для удаления мячей}]{code/Deleter.cs}

Скрипт \texttt{Deleter.cs} отвечает за удаление мячей при удалении от игрока.

\subsection{Логика игры}

Код управления сохранением:
\lstinputlisting[language=Python,
	caption=\leftline{Код управления сохранениемв}]{code/SaveLoadManager.cs}

Скрипт \texttt{SaveLoadManager.cs} отвечает за сохранение и загрузку данных о рейтинге.

Код управления меню:
\lstinputlisting[language=Python,
	caption=\leftline{Код управления меню}]{code/menu.cs}

Скрипт \texttt{menu.cs} отвечает за меню.


